\documentclass[aps]{revtex4-2}
\usepackage[margin=1in]{geometry}
\usepackage{graphicx}
\usepackage{amsmath, amssymb}
\usepackage{siunitx}
\usepackage{booktabs}
\usepackage{caption}  
\usepackage{xurl}      
\usepackage[hidelinks]{hyperref}   
\usepackage{subcaption}
\usepackage{enumitem}
\graphicspath{{images/}}
\usepackage{lipsum}
\usepackage{amsmath,amssymb}
\usepackage{url}
\usepackage{pgfgantt}
\usepackage{placeins}
\usepackage{appendix}

\hypersetup{
  colorlinks=true,
  linkcolor=blue,
  citecolor=teal,
  urlcolor=blue
}

% Title


\usepackage{float}
\begin{document}

  \title{Final Report: An Extended Maxwell's Refrigerator }
\author{Erdem Pülat}
\affiliation{Bilkent University, Department of Physics}
\date{\today}

\begin{abstract}
This report documents my reproduction and extension of the autonomous Maxwell demon model by Mandal \emph{et al.}\cite{mandal2013refrigerator}, a minimal device that rectifies thermal transitions while processing information on a moving tape. First, the original two-state ``Maxwell refrigerator'' is numerically reproduced with the help of a Gillespie simulation, and its known limits and phase behavior are inspected. Then, we go on to demonstrate an alternative way of refrigeration without slowing the tape: stacking multiple demons in series, where increasing the number of demons  mimics increasing the interaction time for a single demon. Afterward, the framework is generalized to demons with $n>2$ internal states arranged as energy ladders, clarifying how model choices about energy scales and transition parameters affect the interpretation of $\Delta E$. Finally, a two-bit interacting demon that processes bit-pairs and can actively reshape correlations on the tape is introduced, extending the information-thermodynamics interplay beyond independent bits to correlated bits.
\end{abstract}
\maketitle




\section{Introduction}
\label{sec:intro}

In a 1871 thought experiment, Maxwell imagined a ``neat-fingered being'' that had such information about the system that it could sort gas molecules by speed, creating a temperature difference between two chambers without work \cite{leff2003maxwell}. This paradox—a seeming violation of the second law of thermodynamics on the surface—is, in truth, a wonderful thought experiment that connects information and entropy. Any physical device performing such sorting must process and store information about molecular states. The process of writing this information to any sort of memory increases the Shannon entropy of the memory, compensating for the decrease in thermodynamic entropy in other places. Thus, we arrive at a balance, and the total entropy is still ever-increasing.

'Maxwell's demon' can mean any setup that uses microscopic fluctuations to lower thermodynamic entropy\cite{maxwell1871theory,szilard1929}. These models/mechanisms obey the second law if the device is also "writing" information to a memory: the rise in information (Shannon) entropy offsets the thermodynamic entropy drop. If that information is later erased, Landauer's principle says that thermodynamic entropy must increase elsewhere\cite{landauer1961}. Counting Shannon and Clausius entropies together ensures that the system obeys the second law as long as their sum remains constant or decreases. It is essential to note that this approach links information entropy to the second law of thermodynamics.

Mandal \emph{et al.} \cite{mandal2013refrigerator} made this abstract principle more concrete by designing a simple model: a classical two-state system (a ``demon'') that interacts with thermal baths at two temperatures while sequentially processing bits on a moving tape (FIG \ref{fig:demondesign}).  When the incoming tape has an excess of $0$ bits, the demon can exploit this bias to pump heat from a cold reservoir to a hot one, similar to a refrigerator; hence, the name Maxwell's refrigerator.\cite{mandal2013refrigerator} Alternatively, when the temperature gradient opposes the tape bias, the demon erases information from the tape, reducing its Shannon entropy at the thermodynamic cost of allowing heat to flow downhill.
\begin{figure}[h]
    \centering
    \includegraphics[width=.7\linewidth]{images/demon_design.png}
    \caption{(a) The demon processes one bit at a time while exchanging energy with two heat baths.
(b) It changes its own state via the hot bath (vertical arrows) and, together with the current bit, makes coupled transitions 0d $\leftrightarrow$1u via the cold bath (diagonal arrows). }\cite{mandal2013refrigerator}
    \label{fig:demondesign}
\end{figure}

My work has three objectives: (i) to examine the model's parameters and behavior and ensure that a numerical implementation conforms to the analytic results, (ii) to explore extensions that improve performance without changing the reservoir temperatures---in particular stacking demons and modifying the demon's state space---and (iii) to study a two-bit interacting variant in which we take a closer look to correlations. We find that stacking demons can replicate the effect of slowing down the tape, that increasing the number of demon's internal states is dependent on the choices of energy differences, and that two-bit interactions can reshape correlations on the tape in nontrivial ways, and the thermal baths have great importance.

\section{Two-State Demon: Reproduction of the Maxwell's Refrigerator}
\label{sec:two_state}

\subsection{Setup and Operating Mechanism}

The demon is a two-state system with states $u$ (up) and $d$ (down) separated by energy $\Delta E = E_u - E_d > 0$. It sits next to a tape of bits (each bit has states $0$ and $1$ at equal energies) that moves past at constant velocity $v$. The demon interacts with each bit for a fixed time $\tau$ before the tape advances to the next bit. Crucially, the demon simultaneously contacts two thermal reservoirs: a hot bath at temperature $T_H$ and a cold bath at temperature $T_C < T_H$ \cite{mandal2013refrigerator}.

 The demon can change its internal state in two ways. It can flip between $u$ and $d$ by absorbing or emitting energy $\Delta E$ to the hot bath, called``intrinsic'' transitions. If the demon is in state $d$ and the current bit shows $0$, they can also together transition to $1u$: the bit flips to $1$ and the demon jumps to $u$, with the energy $\Delta E$ coming from the cold bath. The reverse ($1u \to 0d$) is also possible, returning energy to the cold bath. Importantly, no other cooperative transitions are allowed—you cannot go from $0u$ to $1d$\cite{mandal2013refrigerator}.

This restricted coupling creates a rectification effect. Consider an incoming tape with mostly $0$ bits. When the demon encounters a $0$, there's a chance to execute a $0d \to 1u$ transition, drawing energy $\Delta E$ from the cold bath. The demon might then go back to state $d$ via the hot channel, dumping $\Delta E$ into the hot bath, as such energy flows from cold to hot, with the outgoing tape showing more $1$ bits as a record/result of this process. The incoming bias (excess of $0$'s) is similar to a thermodynamic ``fuel'' that enables refrigeration.

\subsection{Transition Rates and Detailed Balance}

To make this quantitative, we assign rates to all allowed transitions, with respect to the detailed balance condition. The hot-channel transitions must satisfy:
\begin{equation}
\frac{R_{d\to u}}{R_{u\to d}} = e^{-\beta_H \Delta E}, \quad \text{where} \quad \beta_H = \frac{1}{k_B T_H}.
\end{equation}
Following \cite{mandal2013refrigerator}, 
\begin{equation}
R_{d\to u} = \gamma(1-\sigma), \quad R_{u\to d} = \gamma(1+\sigma)
\label{eq:hot_rates}
\end{equation}

\begin{equation}
    \sigma = \tanh\left(\frac{\beta_H \Delta E}{2}\right)
\end{equation}
where $\gamma > 0$ sets how fast the demon equilibrates with the hot bath.

Similarly, the cold-channel cooperative transitions obey detailed balance at $T_C$:
\begin{equation}
\frac{R_{0d\to 1u}}{R_{1u\to 0d}} = e^{-\beta_C \Delta E}, \quad \text{where} \quad \beta_C = \frac{1}{k_B T_C}.
\end{equation}
We set
\begin{equation}
R_{0d\to 1u} = 1-\omega, \quad R_{1u\to 0d} = 1+\omega.
\label{eq:cold_rates}
\end{equation}

\begin{equation}
    \omega = \tanh\left(\frac{\beta_C \Delta E}{2}\right)
\end{equation}

\cite{mandal2013refrigerator}


The temperature difference determines which bath ``wins'' the tug-of-war to push either $0d \to 1u$ or $1u \to 0d$. Mandal \emph{et al.} introduces 
\begin{equation}
\epsilon = \frac{\omega - \sigma}{1 - \omega\sigma} = \tanh\left[\frac{(\beta_C - \beta_H)\Delta E}{2}\right]
\label{eq:epsilon}
\end{equation}
which represents the temperature difference, and we will later see that this variable will set a limit for the output tape bias.

\subsection{Information Current and Heat Flow}

Let the incoming tape have bias $\delta \equiv p_0 - p_1$ (the excess fraction of $0$ bits). After interacting with the demon, bits emerge with bias $\delta' = p'_0-p'_1$. We define the $\Phi$, information current:
\begin{equation}
\Phi \equiv p'_1 - p_1 = \frac{\delta - \delta'}{2}.
\label{eq:Phi}
\end{equation}
$\Phi$ measures the net rate at which $0$ bits flip to $1$. The name ``information current'' reflects that $\Phi > 0$ means the demon is writing information\cite{mandal2013refrigerator}.

 Each net $0 \to 1$ flip corresponds to one cooperative transition $0d \to 1u$. Since that transition extracts energy $\Delta E$ from the cold bath, the average heat flow from cold to hot per bit interaction is
\begin{equation}
Q_{c\to h} = \Phi \cdot \Delta E.
\label{eq:Qch}
\end{equation}
Thus, a positive $\Phi$  means refrigeration: heat flows against the gradient.

The Mandal et. al. solution gives
\begin{equation}
\Phi = \frac{\delta - \epsilon}{2} \eta(\gamma, \tau, \sigma, \omega, \delta), \quad \eta > 0,
\label{eq:Phi_solution}
\end{equation}
where $\eta$ depends on all parameters but is strictly positive, proof is in the appendix of \cite{mandal2013refrigerator}. Consequently, the sign of $\Phi$ is determined by the difference between the incoming bias $\delta$ and the thermal parameter $\epsilon$. When $\delta > \epsilon$, the tape bias wins and we get refrigeration. When $\delta < \epsilon$, the temperature gradient wins and the demon acts as an eraser, leading to a $\delta'$ closer to $\epsilon$ than $\delta$ \cite{mandal2013refrigerator}.


To take into account the second law,  we look at the modified Clausius inequality
\begin{equation}
Q_{c\to h}(\beta_H - \beta_C) + \Delta S_B \geq 0,
\label{eq:second_law}
\end{equation}
where $\Delta S_B = S(\delta') - S(\delta)$ is the change in Shannon entropy per bit, with
\begin{equation}
S(\delta) = -\frac{1-\delta}{2}\ln\frac{1-\delta}{2} - \frac{1+\delta}{2}\ln\frac{1+\delta}{2}.
\end{equation}
This states that the thermodynamic entropy decrease (first term, negative when heat flows up the gradient) must be compensated by an increase in information entropy.


\subsection{Numerical Implementation}

Here, it is important to note that the original paper \cite{mandal2013refrigerator} does not contain a numerical implementation of the problem; rather, its main achievement is the analytic solution of this setup. This enables us to cross-check our numerical implementation with the analytical solution. 

\subsubsection{Gillespie Method}
We employ the stochastic simulation algorithm (SSA) of Gillespie~\cite{gillespie1976} inside each interaction window of duration \(\tau\), during which a single bit interacts with the demon and the thermal reservoirs.\footnote{Initially, I attempted to simulate multiple extremely small $dt$ windows at each $\tau$; however, this proved to be less reliable and significantly more computationally expensive. As such, I decided to only use the Gillespie Method going forward.}

For our joint state \(s\in\{0u,0d,1u,1d\}\), the total exit rate is
\[
r_{\text{out}}(s)=
\begin{cases}
\gamma(1+\sigma), & s=0u,\\[2pt]
\gamma(1-\sigma)+(1-\omega), & s=0d,\\[2pt]
\gamma(1+\sigma)+(1+\omega), & s=1u,\\[2pt]
\gamma(1-\sigma), & s=1d~.
\end{cases}
\] Thanks to Eq.~ \ref{eq:hot_rates} \& \ref{eq:cold_rates}


Given the current joint state \(s\) inside the window, let \(\{r_j\}\) be the rates of the reactions that can fire from \(s\) (one or two transition channels, depending on \(s\)). The SSA's waiting time is
\[
dt \sim \mathrm{Exp}\!\big(r_{\text{out}}(s)\big),\quad\text{i.e.}\quad dt=-\frac{\ln u_1}{r_{\text{out}}(s)},\; u_1\sim\mathcal U(0,1),
\]
and, conditional on a jump, picks channel \(j\) with probability \(r_j/r_{\text{out}}(s)\) using a second uniform \(u_2\).


For each bit–demon encounter:
\begin{enumerate}
\item Initialize \(t\leftarrow 0\) with incoming bit \(b\in\{0,1\}\) (drawn from \(p_0,p_1\)) and demon state \(d\in\{u,d\}\). Set joint state \(s=(b,d)\).
\item \textbf{While} \(t<\tau\):
  \begin{enumerate}
  \item Compute all enabled rates \(\{r_j\}\) from \(s\) and their sum \(r_{\text{out}}(s)\).
  \item Draw \(dt=-\ln u_1/r_{\text{out}}(s)\). If \(t+dt\ge\tau\): set \(t\leftarrow\tau\) and \textbf{break} (no jump occurs in this window).
  \item Otherwise, select the firing channel \(j\) by finding the smallest index with
  \[
  \sum_{k\le j-1} r_k < u_2\, r_{\text{out}}(s) \le \sum_{k\le j} r_k,\qquad u_2\sim\mathcal U(0,1).
  \]
  \item Update the joint state \(s\) according to \(j\), advance time \(t\leftarrow t+dt\).
 \end{enumerate}
\end{enumerate}



Additionally, for our numerical implementation, it is more convenient to make use of the dimensionless parameters $\sigma$, $\omega$, and $\epsilon$ going forward, and focus on the output parameter $\Phi$. $\Phi$ defines the \emph{net} number of $0\to1$ flips per bit, and each  $0\to1$ flip is an action against the thermal gradient, a refrigeration, and vice versa is also true for the eraser case ($\Phi<0$). Thus, it quantifies how close the system is to a perfect refrigerator; if $\Phi = 1$, all bit flips are $ 0 \ to 1$, which would mean that the heat flow is only from the cold reservoir to the hot reservoir. However, this would only be possible with $\epsilon = 0$,(See Eq. \ref{eq:Phi_solution}) which is frankly a boring system with $\sigma = \omega \rightarrow T_H = T_C$. Which is why our aim in this project is to get as close as possible to $\Phi = 1$ without changing the temperature differences or $\Delta E$. 

\subsubsection{Results}
With that out of the way, I first attempted to reproduce the phase diagram presented in the original paper using the aforementioned method.
\begin{figure}[h]
  \centering
  \begin{subfigure}{0.48\linewidth}
    \includegraphics[width=\linewidth]{two_state_diagrams/diagram_1.png}
    \caption{}\label{fig:p1}
  \end{subfigure}\hfill
  \begin{subfigure}{0.48\linewidth}
    \includegraphics[width=\linewidth]{two_state_diagrams/diagram_2.png}
    \caption{}\label{fig:p2}
  \end{subfigure}

  \vspace{0.6em}

  \begin{subfigure}{0.48\linewidth}
    \includegraphics[width=\linewidth]{two_state_diagrams/diagram_3.png}
    \caption{}\label{fig:p3}
  \end{subfigure}\hfill
  \begin{subfigure}{0.48\linewidth}
    \includegraphics[width=\linewidth]{two_state_diagrams/paper_diagram.png}
    \caption{}\label{fig:p4}
  \end{subfigure}

  \caption{a, b, and c are my simulation results, while d is the exact solution in the paper. Black lines represent $\Delta S_B = 0$. The values on the lines in image d represent $\tau$ values, and plots a, b, c represent $\tau = 1.0, 4.0, 10000$ respectively, with blue areas acting as erasers and red parts representing refrigeration. Also all plots have $\omega = 1/2, \gamma = 0.1$}
  \label{fig:pgrid}
\end{figure}
\footnote{Throughout this section, $n=2$ denotes the number of internal demon states.}
As shown in FIG. \ref{fig:pgrid}, the simulation results are consistent with the phase diagram presented in the original paper (FIG. \ref{fig:p4}). The next part that I wanted to check was the $\gamma \to \infty$ limit, as the original paper states that in that limit, the expression for $\Phi$ simplifies to:
\begin{equation}
    \Phi = \frac{\delta - \epsilon}{2} [1-e^{-(1-\sigma \omega )\tau}].
\end{equation} \cite{mandal2013refrigerator}

Thus, we can check if we approach this limit with high gamma. 
\begin{figure}
  \centering
    \begin{subfigure}{0.75\linewidth}
    \includegraphics[width=\linewidth]{gamma/gamma_vs_phi_higher_tau_3.png}
    \caption{}\label{fig:ghtau3}
  \end{subfigure}\hfill
  \begin{subfigure}{0.75\linewidth}
    \includegraphics[width=\linewidth]{gamma/gamma_vs_phi_higher_tau_1.png}
    \caption{}\label{fig:ghtau1}
  \end{subfigure}\hfill
  \begin{subfigure}{0.75\linewidth}
    \includegraphics[width=\linewidth]{gamma/gamma_vs_phi_higher_tau_2.png}
    \caption{}\label{fig:ghtau2}
  \end{subfigure}\hfill

  \caption{Here we go through gamma values [1-100] and plot out the $\Phi$ values. We observe the desired behavior and approach the theoretical limit very fast, thanks to the high  $\tau = 5.0$. Note the different $\sigma$ and $\omega$ values for each plot, which are mainly for confirmation.}
  \label{fig:gamma1}
\end{figure}
\begin{figure}
  \centering
    \begin{subfigure}{0.75\linewidth}
    \includegraphics[width=\linewidth]{gamma/gamma_vs_phi_sigmaomega_3.png}
    \caption{}\label{fig:ghtau6}
  \end{subfigure}\hfill
  \begin{subfigure}{0.75\linewidth}
    \includegraphics[width=\linewidth]{gamma/gamma_vs_phi_sigmaomega_1.png}
    \caption{}\label{fig:ghtau4}
  \end{subfigure}\hfill
  \begin{subfigure}{0.75\linewidth}
    \includegraphics[width=\linewidth]{gamma/gamma_vs_phi_sigmaomega_2.png}
    \caption{}\label{fig:ghtau5}
  \end{subfigure}\hfill

  \caption{Here we go through gamma values [1-100] and plot out the $\Phi$ values but approach the theoretical limit slower than the $\tau = 5.0$ case, since here $\tau = 1.0$. This is a pattern we will later see again.}
  \label{fig:gamma2}
\end{figure}

Figures \ref{fig:gamma1} and \ref{fig:gamma2} show that the numerical simulation gets within the theoretical limit with each parameter set. The other pattern we observe is that there might be a positive correlation with $\tau$ and $\Phi$. 
\begin{figure}[H]
  \centering
  \begin{subfigure}{0.75\linewidth}
    \includegraphics[width=\linewidth]{phivstau/phi_vs_tau_1.png}
    \caption{}\label{fig:phivtau1}
  \end{subfigure}\hfill
  \begin{subfigure}{0.75\linewidth}
    \includegraphics[width=\linewidth]{phivstau/phi_vs_tau_2.png}
    \caption{}\label{fig:phivtau2}
  \end{subfigure}\hfill
  \begin{subfigure}{0.75\linewidth}
    \includegraphics[width=\linewidth]{phivstau/phi_vs_tau_3.png}
    \caption{}\label{fig:phivtau3}
  \end{subfigure}\hfill
  \caption{Here we sweep through $\tau$ values between 1.0- 100.0 and see both how increasing $\tau$ leads to increasing $\Phi$. The interesting point is that we arrive at the same theoretical ceiling as before, without setting $\gamma \to \infty$, $\gamma = 1.0$ throughout a, b, c. The physical parameters determine the limit again.}
  \label{fig:phivtaugrid}
\end{figure}

\begin{figure}[H]
  \centering
  \begin{subfigure}{0.75\linewidth}
    \includegraphics[width=\linewidth]{N/length_vs_phi_1.png}
    \caption{}\label{fig:Nvsphi1}
  \end{subfigure}\hfill
  \begin{subfigure}{0.75\linewidth}
    \includegraphics[width=\linewidth]{N/length_vs_phi_2.png}
    \caption{}\label{fig:Nvsphi2}
  \end{subfigure}\hfill
  \caption{Here we go through N values from 200 to 80000. I did split the plots so that the latter part is clearer to the eye, and we can see that the $\Phi$ value stabilizes around $N=10000$}
  \label{fig:Nvsphigrid}
\end{figure}

Looking at the FIG. \ref{fig:phivtaugrid} shows that we reach the same theoretical limits as in FIG. \ref{fig:gamma1}-\ref{fig:gamma2} when we check for the same $\omega$ and $\sigma$ values. This reveals that $\tau$ and $\gamma$ can lead to an increase in $\Phi$ up to a point determined by $\omega$ and $\sigma$ values. Lastly, the relationship between the tape's length and $\Phi$ can also be seen in FIG. \ref{fig:Nvsphigrid}, which is kind of expected and does not show any reason to go for a bigger $N$ than 10000. However, I chose a smaller $N=4000$ and sacrificed accuracy for speed, as the main aim of this section was primarily to observe patterns.

\subsection{Stacked Demons}
In the previous section we saw that  increasing $\tau$ (or pushing $\gamma$ large) pushes the demon closer to its ceiling, but at a physical cost: a larger $\tau$ corresponds to a slower tape, as each bit must remain coupled to the demon for longer.

An alternative is to place multiple identical demons in series along the tape's path. In this ``stacked'' design, each bit interacts with demon 1 for time $\tau$, then demon 2 for time $\tau$, and so on, for a total of $K$ demons. The tape does not need to slow down; instead, the physical device becomes longer. In practice this gives us an effective interaction time $\tau_{\mathrm{eff}} \approx K\tau$, which works the same way as increasing $\tau$ for a single demon.

FIG.~\ref{fig:stackedKvsTau} compares the information current for a single demon with interaction time $\tau_{\mathrm{eff}}$ to the stacked case with fixed per-demon $\tau$. The close agreement shows that increasing $K$ reproduces the same trend as increasing $\tau$.

\begin{figure}[H]
  \centering
  \includegraphics[width=0.7\linewidth]{images/stacked_k_vs_tau.png}
  \caption{Comparison of increasing $\tau$ for a single demon versus stacking $K$ demons with fixed per-demon interaction time , plotted against the effective interaction time $\tau_{\mathrm{eff}}=K\tau$. Here, $\sigma = 0.3$, $\omega = 0.8$, and $\delta = 1.0$.}
  \label{fig:stackedKvsTau}
\end{figure}

\begin{figure}[H]
  \centering
  \includegraphics[width=0.7\linewidth]{images/phi_vs_K_stacked.png}
  \caption{Here we have the same previous case as FIG.~\ref{fig:stackedKvsTau} but the stacked demons are plotted alone.}
  \label{fig:phiVsKStacked}
\end{figure}

In conclusion, varying $\tau$ is important to show how the two-state demon approaches its ceiling and redesigns of the architecture with the help this fact was shown to be impactful, and is also fun. Stacking is one such redesign that preserves the rules with a trivial addition, and can even be count as we don't need to slow down the tape.
  
\section{N-State Demon: Generalization to Energy Ladders}
\label{sec:n_state}

The two-state demon has only one energy step $\Delta E$ to mediate the transition. A natural question is: would a demon with a different internal structure—more energy levels—perform better? Can we get a system that gets as close to a perfect refrigerator as physically can this way?


We can generalize the demon to $n$ states arranged on an energy ladder: $E_0 < E_1 < \cdots < E_{n-1}$ with spacing $\Delta E_k = E_{k+1} - E_k$.  The joint system now has $2n$ states. The key design choice is how to couple the ladder to bit flips. Generalizing the previous interactions fits our needs: a cooperative transition flips the bit from $0$ to $1$, i.e., $(b=0, s_k) \leftrightarrow (b=1, s_{k+1})$ for $k = 0, \ldots, n-2$. This preserves the rectification structure: you cannot flip to a $1$ bit without raising the demon's energy, and you cannot flip to a $0$ bit without lowering it.


The hot and cold transition channels become:

Hot channel: Transitions $s_i \leftrightarrow s_j$ within each bit sector ($b=0$ or $b=1$) obey detailed balance at $T_H$:
\begin{equation}
\frac{R_{s_i \to s_{i+1}}}{R_{s_{i+1} \to s_i}} = e^{-\beta_H (E_{i+1} - E_i)}.
\label{eq:firt}
\end{equation}
 I use the same parametrization as before: $R_{s_k \to s_{k+1}} = \gamma(1 - \sigma_k)$ and $R_{s_{k+1} \to s_k} = \gamma(1 + \sigma_k)$, where now $\sigma_k = \tanh(\beta_H \Delta E_k/2)$ depends on the local spacing.

Cold channel: Cooperative transitions $(b=0, s_k) \leftrightarrow (b=1, s_{k+1})$ obey detailed balance at $T_C$:
\begin{equation}
\frac{R_{0s_k \to 1s_{k+1}}}{R_{1s_{k+1} \to 0s_k}} = e^{-\beta_C \Delta E_k},
\label{eq:secnt}
\end{equation}
parametrized as $R_{0s_k \to 1s_{k+1}} = 1 - \omega_k$ and $R_{1s_{k+1} \to 0s_k} = 1 + \omega_k$ with $\omega_k = \tanh(\beta_C \Delta E_k/2)$.

For a uniform ladder ($\Delta E_k = \Delta E/(n-1)$ for all $k$), the parameters $\sigma_k$ and $\omega_k$ become constant, recovering the two-state $\epsilon = (\omega - \sigma)/(1 - \omega\sigma)$ as the effective asymmetry. Note that plots below will not have $\sigma_k$ and $\omega_k$. The modified Clausius inequality \eqref{eq:second_law} continues to hold, though the functional form of $\eta$  now depends on $N$ and the ladder profile, and may be unattainable.

\subsection{Numerical Implementation}

We can reuse the previous SSA method by tracking the $2N$-state joint system and its corresponding transition rates in equations \ref{eq:firt} and \ref{eq:secnt}.


\subsubsection{Results: Number of Demon States vs $\Phi$}

\begin{figure}[H]
    \centering
    \includegraphics[width=0.6\linewidth]{demonn/Demon_Result.png}
    \caption{Here we have $\omega=0.7$, $\sigma=0.2$. $\tau = 10.0$ to hit the aforementioned limit earlier. We see a clear rise in $\Phi$ as we increase the size of the energy ladder.}
    \label{fig:demonRes}
\end{figure}

\begin{figure}[H]
    \centering
    \includegraphics[width=0.6\linewidth]{demonn/demon_n_vs_phi_2.png}
    \caption{For this one, we continue to increase the size of the ladder, but not much change happens after $n=20$.}
    \label{fig:demonRes2}
\end{figure}
FIG.~\ref{fig:demonRes} and FIG.~\ref{fig:demonRes2} show that increasing the number of demon states $n$ leads to a higher information current $\Phi$, approaching closer to the perfect refrigerator limit of $\Phi = 1$. However, the returns diminish with larger $n$, and performance seems to plateau beyond $n \approx 20$.

At this point, it becomes important to be explicit about what is meant by ``keeping the parameters the same'' in an $n$-state ladder. One can either (i) hold reservoir temperatures $T_H,T_C$ fixed while changing the per-rung energy spacing $\Delta E_k$, or (ii) hold the \,dimensionless\, transition parameters---the quantities that directly enter the transition rates, $\sigma$ and $\omega$---fixed for each level.

Option (i) is the one employed to produce plots FIG.~\ref{fig:demonRes} and FIG.~\ref{fig:demonRes2}. It is physically natural if one insists on the same reservoirs, but it also changes $\sigma_k=\tanh(\beta_H\Delta E_k/2)$ and $\omega_k=\tanh(\beta_C\Delta E_k/2)$ because of the way the ladder is refined, because of a lower $\Delta E$. This lower $\Delta E$ means that each cooperative transition exchanges less energy with the baths, making it easier for the demon to drive $0 \to 1$ flips against the thermal gradient. Thus, part of the performance gain in FIG.~\ref{fig:demonRes} and FIG.~\ref{fig:demonRes2} comes from effectively reducing the thermal bias as $n$ increases, and can be taken as inefficient since the energy transfer per bit is much less than if we'd keep $\sigma$ and $\omega$ fixed.

Option (ii) keeps $\sigma$ and $\omega$ fixed \,per rung\,, so that each adjacent transition is parameterized the same way as the two-state demon. In that case the ladder extension is a structural change rather than a hidden change of thermal bias or $\Delta E$. FIG.~\ref{fig:nSigmaOmega} shows $\Phi$ versus $n$ under this ``fixed-$\sigma,\omega$ per level'' convention. Here, $\Delta E$ between states are constant for all demons, so increasing $n$ increases the overall energy span of the ladder. No performance gain is observed beyond $n=2$ in this case, indicating that simply enlarging the state space without changing the dimensionless transition biases does not help.

\begin{figure}[H]
  \centering
  \includegraphics[width=0.6\linewidth]{images/n_state_phi_vs_n_sigmaomega.png}
  \caption{$\Phi$ versus number of demon states $n$ for a laddered demon where each ladder step uses the same dimensionless transition biases ($\sigma$ and $\omega$) as the baseline two-state case. This isolates the effect of enlarging the state space from the separate effect of changing $\Delta E$ by shrinking energy spacings at fixed temperatures.}
  \label{fig:nSigmaOmega}
\end{figure}

Taken together, these results show that $\Delta E$ requires special care in our discussion: in an $n$-state ladder there is a distinction between a per-rung energy gap and the overall energy span of the ladder. And the result is that $\Delta E$ is the main factor that can affect performance, rather than simply enlarging the state space. However, this is not our aim, as mentioned before, we do not want to mess with physical parameters. Thus, enlarging the state space alone does not help much, and would be only helpful in the case of an infinitely long tape, which could refrigerate on and on, with a low $\Delta E$, with the expense of the tape.

\section{Two-Bit Interacting Demon}
\label{sec:two_bit}

The original Maxwell refrigerator processes bits independently: the demon's states couple to a single bit and the only "memory" effect across time is the demon's own internal state. Another interesting extension to our original model \cite{mandal2013refrigerator} is to allow the demon to interact with \emph{pairs} of bits. We have the following motivations for this: First, it introduces a minimal notion of structured input beyond a simple bias parameter, since the two-bit correlations in the incoming bit would gain meaning, and it allows cooperative transitions that can create or destroy those correlations while still respecting detailed balance with the reservoirs. Second, it allows us to explore how the baths influence correlations in the tape, rather than just single-bit statistics. Here we want to see how the two designs interact with correlations and compare and contrast them. However, this section will be a little more different from the previous ones, since we won't be focusing on $\Phi$, energy or refrigeration, since we covered both in the previous section. Here we will try to check the demon's ability to modify correlations on the tape while still respecting the second law, pay closer attention the heat baths.

In the two-bit interacting model (TB-Demon) used here, the demon remains a two-state system ($u/d$), but the tape is processed in adjacent pairs ($00,01,10,11$). Intrinsic transitions $u\leftrightarrow d$ exchange energy with the hot bath as before. Cooperative transitions now involve joint changes of the demon state and the current bit-pair, and are constructed to satisfy detailed balance. A key difference is that the TB-demon can modify the pair distribution in ways that wouldn't be possible when interacting with one bit at a time.

\subsection{Structure and Allowed Transitions}
\begin{figure}[H]
  \centering
  \includegraphics[width=0.6\linewidth]{demonn/two-bit-state-structure.png}
  \caption{State diagram for the two-bit interacting demon. The demon has two internal states ($u$ and $d$) and we have pairs of bits from the tape ($00,01,10,11$). Intrinsic transitions (vertical ones) flip the demon state without changing the bit-pair, while cooperative transitions (diagonal ones) change both the demon state and the bit-pair via the cold bath.}
  \label{fig:twoBitDemonStates}
  
\end{figure}
\begin{figure}[H]
  \centering
  \includegraphics[width=0.6\linewidth]{demonn/two_bit_demon_design.png}
  \caption{An overview of our deisgn with the two-bit interacting demon. The demon interacts with pairs of bits from the tape, allowing it to modify correlations between bits while exchanging energy with the thermal reservoirs.}
  \label{fig:twoBitDemonStates}
\end{figure}
The two-bit interacting demon still has two internal states, $u$ (up) and $d$ (down). The difference is in the tape: at each interaction window the demon couples to a pair of adjacent bits, so the tape microstate for that window is one of $\{00,01,10,11\}$. The joint configuration space for a single interaction is therefore
\[
\{d00,d01,d10,d11,u00,u01,u10,u11\},
\]
eight states in total.

As in the single-bit model, we have two types of transitions:
Intrinsic transitions (hot bath): the demon flips without changing the bit-pair,
\(uXY\leftrightarrow dXY\) for any \(XY\in\{00,01,10,11\}\).

Cooperative transitions (cold bath): only specific coupled flips are allowed, namely
\begin{align}
  d00 &\leftrightarrow u01,\\
  u10 &\leftrightarrow d11.
\end{align}

Here you'll notice that for the second cooperative transition flips, when the second bit of the pair goes from $0$ to $1$ the demon goes from $u$ to $d$, which is the opposite of any of the previous transitions. This is because we focus on the correlation impact rather than refrigiration now. 
These cooperative moves make the structure concrete: the demon can only change the pair by flipping the second bit, and only in a demon-state-dependent way (from $00$ to $01$ when going $d\to u$, and from $10$ to $11$ when going $u\to d$). All other pair changes are forbidden, similar to the constraints of the original model and ensuring that heat exchange with the cold bath occurs only through these designated coupled transitions. The transition rates are parametrized similarly to before, with intrinsic rates obeying detailed balance at $T_H$ and cooperative rates obeying detailed balance at $T_C$, and our dear dimensionless variables $\sigma$, $\omega$, and $\epsilon$ defined as before.

\subsection{Results}
FIG.~\ref{fig:twoBitDemonCorr} compares the two-bit demon to the single-bit demon on the same initial tape. Besides the information current $\Phi$ and the tape fraction changes ($p_0, p_1$), we track the change in pair correlation and the change in mutual information between the two bits in each pair. These results show that the two-bit demmon's ability to interact with two-bit correlations make it so that the heat baths shape the two-bit correlations. FIG.~\ref{fig:twoBitDemonCorr2} reinforces this idea as with a lower temperature difference, we see much less 2-bit correlation for the TBDemon compared to FIG.~\ref{fig:twoBitDemonCorr}.  As such, we see that in the first case TBDemon is not able to lower the correlation to a completely uncorrelated and random state, because the bath's gradient resists it. In contrast, the original demon can reduce two-bit correlations more because the heat baths cannot resist it as they do not interact with two-bit correlations. The single-bit demon reduces these correlations more effectively than the two-bit demon, which is constrained by its interactions with the baths.

\begin{figure}
  \centering  
  \includegraphics[width=.8\linewidth]{demonn/two-bit-comp-plot.png}
  \caption{Here we have the most interesting case with an input tape that has a high level of two-bit correlations. We see here how we get the unexpected: one-bit demon reduces correlations more than the two-bit demon, as the two-bit demon's interactions with the baths resist correlation erasure.}
  \label{fig:twoBitDemonCorr}
\end{figure}

\begin{figure}
  \centering  
  \includegraphics[width=.8\linewidth]{demonn/two-bit-comp-plot-lesstempdiff.png}
  \caption{Here we have another case of checking correlation changes, but with a smaller temperature difference between the baths. The difference is apperant, as the two bit demon now is able to reduce correlations more thanks to less resistance from the baths.}
  \label{fig:twoBitDemonCorr2}
\end{figure}



\subsection{Refrigeration Perspective}
Here I'll briefly discuss the refrigeration perspective of the two-bit interacting demon. We will look into a modified version of our TBDemon, where we change the cooperative transitions to:
\begin{align}
  d00 &\leftrightarrow u01,\\
  d10 &\leftrightarrow u11.
\end{align}


\begin{figure}
  \centering  
  \includegraphics[width=.8\linewidth]{demonn/two-bit-comp-plot-refrig.png}
  \caption{Here we have the refrigeration performance of the two-bit interacting demon with modified cooperative transitions. You'll notice that the performance is much less than the original demon, we have a much weaker conversion of 0's to 1's.}
  \label{fig:twoBitDemonRefrig}
\end{figure}

As a result, the $0$ states will be lower in energy than the $1$ states, similar to the original model. Thus, we can check for refrigeration performance as before. FIG.~\ref{fig:twoBitDemonRefrig} shows that our new demon is much less effective. This is because of the allowed transitions: the demon can only flip the second bit of the pair, which limits its ability to drive $0 \to 1$ flips against the thermal gradient. The cooperative transitions are more constrained than in the single-bit model, reducing the demon's capacity to perform refrigeration effectively. 

\section{Conclusion}
\label{sec:conclusion}
We reproduced the autonomous Maxwell refrigerator of Mandal \emph{et al.}~\cite{mandal2013refrigerator} with the help of a Gillespie simulation and confirmed that the numerical results agree with the known analytic solution of the model. In particular, the physical parameter impacts and limiting trends are similar, as such we can say that the simulations remain consistent with the expected information–thermodynamics balance.

We then explored possible modifications in order to get improved refrigeration performance without changing the reservoir temperatures. First, stacking multiple identical demons in series was shown to effectively mimic a longer interaction time while keeping the tape speed fixed, offering a practical architectural alternative to slowing the tape down. Second, we generalized the demon to an energy-ladder internal structure and clarified that apparent performance gains depend on what is held fixed when increasing the number of states (physical energy gaps versus dimensionless transition parameters). This distinction is essential for examining improvements fairly.

Finally, we introduced a two-bit interacting demon that operates on bit pairs rather than independent bits. This extension was instrumental in showing how our system deal with correlations and what it would take to reduce them. The result: the baths' influence on correlations limits the demon's ability to erase them, contrasting with the single-bit demon that can reduce correlations more freely. Not focusing on correlation erasure, we modified the two-bit demon to perform refrigeration, but its constrained cooperative transitions limited its effectiveness compared to the original model.

Overall, these extensions show that better refrigeration can be achieved either with redesigns by stacking demons or by changing internal demon structure with a caviat regarding the parameters. Broadening the informational resources available on the tape (multi-bit interactions) shows that whichever correlation the demon is deigned to interact with, will always contain information that will keep it from being in a random state. In all designs, it is apperant that information for the system is not just the bits on the tape, but also the temperatures of the baths. As a result, the information thermodynamics balance remains a guiding principle throughout the designs, and is always can be observed.


\bibliographystyle{unsrt}
\bibliography{refs}
\appendix
\section{Code}\label{app:deriv}
All the code relevant to this project can be found at \href{https://github.com/erdemydt/491_Project}{link}; particularly in the final\_report folder. Simulation.py files could be and was used to generate a bunch of plots relevant to the topic.
\end{document}
