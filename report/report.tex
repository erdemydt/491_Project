\documentclass[aps,twocolumn]{revtex4-2}
\usepackage[margin=1in]{geometry}
\usepackage{graphicx}
\usepackage{amsmath, amssymb}
\usepackage{siunitx}
\usepackage{booktabs}
\usepackage{hyperref}
\usepackage[numbers,sort&compress]{natbib}
\usepackage{caption}
\usepackage{xurl}      
\usepackage[hidelinks]{hyperref} 
\usepackage{subcaption}
\usepackage{enumitem}
\graphicspath{{images/}}
\usepackage{lipsum}
\usepackage{amsmath,amssymb}
\usepackage{url}
\usepackage{pgfgantt}
\usepackage{placeins}
\usepackage{appendix}

\hypersetup{
  colorlinks=true,
  linkcolor=blue,
  citecolor=teal,
  urlcolor=blue
}

% Title


\usepackage{float}
\begin{document}

\title{Progress Report: An Extended Maxwell's Refrigerator }
\author{Erdem Pülat}
\affiliation{Bilkent University, Department of Physics}
\date{\today}

\begin{abstract}
This progress report documents my reproduction and extension of the autonomous Maxwell demon model by Mandal \emph{et al.}\cite{mandal2013refrigerator}, a simple device designed to rectify thermal transitions. The original demon's behavior as a refrigerator is inspected and numerically reproduced. The framework is then generalized to demons with $n > 2$ states and energy ladders, demonstrating that increasing the size of the ladder results in higher refrigeration efficiency and a more efficient demon.  The thermodynamic trade-off between heat flow and information processing persists in the $n$-state case as well.
\end{abstract}
\maketitle




\section{Introduction}
\label{sec:intro}

In a 1871 thought experiment, Maxwell imagined a ``neat-fingered being'' that had such information about the system that it could sort gas molecules by speed, creating a temperature difference between two chambers without work \cite{leff2003maxwell}. This paradox—a seeming violation of the second law of thermodynamics on the surface—is, in truth, a wonderful thought experiment that connects information and entropy. Any physical device performing such sorting must process and store information about molecular states. The process of writing this information to any sort of memory increases the Shannon entropy of the memory, compensating for the decrease in thermodynamic entropy from the heat flow. Thus, we arrive at a balance, and the total entropy is again ever-increasing.

'Maxwell's demon' can mean any setup that uses microscopic fluctuations to lower thermodynamic entropy\cite{maxwell1871theory,szilard1929}. These models/mechanisms obey the second law if the device is also "writing" information to a memory: the rise in information (Shannon) entropy offsets the thermodynamic entropy drop. If that information is later erased, Landauer's principle says that thermodynamic entropy must increase elsewhere\cite{landauer1961}. Counting Shannon and Clausius entropies together ensures that the system obeys the second law as long as their sum remains constant or decreases. It is essential to note that this approach links information entropy to the second law of thermodynamics.

Mandal \emph{et al.} \cite{mandal2013refrigerator} made this abstract principle more concrete by designing a simple model: a classical two-state system (a ``demon'') that interacts with thermal baths at two temperatures while sequentially processing bits on a moving tape (FIG \ref{fig:demondesign}).  When the incoming tape has an excess of $0$ bits, the demon can exploit this bias to pump heat from a cold reservoir to a hot one, similar to a refrigerator; hence, the name Maxwell's refrigerator.\cite{mandal2013refrigerator} Alternatively, when the temperature gradient opposes the tape bias, the demon erases information from the tape, reducing its Shannon entropy at the thermodynamic cost of allowing heat to flow downhill.
\begin{figure}[h]
    \centering
    \includegraphics[width=1\linewidth]{images/demon_design.png}
    \caption{(a) The demon processes one bit at a time while exchanging energy with two heat baths.
(b) It changes its own state via the hot bath (vertical arrows) and, together with the current bit, makes coupled transitions 0d $\leftrightarrow$1u via the cold bath (diagonal arrows). }\cite{mandal2013refrigerator}
    \label{fig:demondesign}
\end{figure}

The present work has two objectives: to examine the model's parameters and behavior, and to ensure that the numerical implementation employed here conforms to the paper's results. Second, extending the model from two to $n$ internal demon states, to see how a larger state space affects the results. 

\section{Two-State Demon: Reproduction of the Maxwell's Refrigerator}
\label{sec:two_state}

\subsection{Physical Setup and Operating Mechanism}

The demon is a two-state system with states $u$ (up) and $d$ (down) separated by energy $\Delta E = E_u - E_d > 0$. It sits next to a tape of bits (each bit has states $0$ and $1$ at equal energies) that moves past at constant velocity $v$. The demon interacts with each bit for a fixed time $\tau$ before the tape advances to the next bit. Crucially, the demon simultaneously contacts two thermal reservoirs: a hot bath at temperature $T_H$ and a cold bath at temperature $T_C < T_H$ \cite{mandal2013refrigerator}.

 The demon can change its internal state in two ways. It can flip between $u$ and $d$ by absorbing or emitting energy $\Delta E$ to the hot bath, called``intrinsic'' transitions. If the demon is in state $d$ and the current bit shows $0$, they can also together transition to $1u$: the bit flips to $1$ and the demon jumps to $u$, with the energy $\Delta E$ coming from the cold bath. The reverse ($1u \to 0d$) is also possible, returning energy to the cold bath. Importantly, no other cooperative transitions are allowed—you cannot go from $0u$ to $1d$\cite{mandal2013refrigerator}.

This restricted coupling creates a rectification effect. Consider an incoming tape with mostly $0$ bits. When the demon encounters a $0$, there's a chance to execute $0d \to 1u$, drawing energy $\Delta E$ from the cold bath. The demon might then go back to $d$ via the hot channel, dumping $\Delta E$ into the hot bath, as such energy flows from cold to hot, with the outgoing tape showing more $1$ bits as a record/result of this process. The incoming bias (excess of $0$'s) is similar to a thermodynamic ``fuel'' that enables refrigeration.

\subsection{Transition Rates and Detailed Balance}

To make this quantitative, we assign rates to all allowed transitions. The hot-channel transitions must satisfy:
\begin{equation}
\frac{R_{d\to u}}{R_{u\to d}} = e^{-\beta_H \Delta E}, \quad \text{where} \quad \beta_H = \frac{1}{k_B T_H}.
\end{equation}
Following \cite{mandal2013refrigerator}, 
\begin{equation}
R_{d\to u} = \gamma(1-\sigma), \quad R_{u\to d} = \gamma(1+\sigma)
\label{eq:hot_rates}
\end{equation}

\begin{equation}
    \sigma = \tanh\left(\frac{\beta_H \Delta E}{2}\right)
\end{equation}
where $\gamma > 0$ sets how fast the demon equilibrates with the hot bath.

Similarly, the cold-channel cooperative transitions obey detailed balance at $T_C$:
\begin{equation}
\frac{R_{0d\to 1u}}{R_{1u\to 0d}} = e^{-\beta_C \Delta E}, \quad \text{where} \quad \beta_C = \frac{1}{k_B T_C}.
\end{equation}
We set
\begin{equation}
R_{0d\to 1u} = 1-\omega, \quad R_{1u\to 0d} = 1+\omega.
\label{eq:cold_rates}
\end{equation}

\begin{equation}
    \omega = \tanh\left(\frac{\beta_C \Delta E}{2}\right)
\end{equation}

\cite{mandal2013refrigerator}


The temperature difference determines which bath ``wins'' the tug-of-war. Mandal \emph{et al.} showed that 
\begin{equation}
\epsilon = \frac{\omega - \sigma}{1 - \omega\sigma} = \tanh\left[\frac{(\beta_C - \beta_H)\Delta E}{2}\right]
\label{eq:epsilon}
\end{equation}
represents the temperature difference, and we will later see that this number will set a limit for the output tape bias.

\subsection{Information Current and Heat Flow}

Let the incoming tape have bias $\delta \equiv p_0 - p_1$ (the excess fraction of $0$ bits). After interacting with the demon, bits emerge with bias $\delta' = p'_0-p'_1$. We define the $\Phi$, information current:
\begin{equation}
\Phi \equiv p'_1 - p_1 = \frac{\delta - \delta'}{2}.
\label{eq:Phi}
\end{equation}
$\Phi$ measures the net rate at which $0$ bits flip to $1$. The name ``information current'' reflects that $\Phi > 0$ means the demon is writing information\cite{mandal2013refrigerator}.

 Each net $0 \to 1$ flip corresponds to one cooperative transition $0d \to 1u$. Since that transition extracts energy $\Delta E$ from the cold bath, the average heat flow from cold to hot per bit interaction is
\begin{equation}
Q_{c\to h} = \Phi \cdot \Delta E.
\label{eq:Qch}
\end{equation}
Thus, a positive $\Phi$  means refrigeration: heat flows against the gradient.

The Mandal et. al. solution gives
\begin{equation}
\Phi = \frac{\delta - \epsilon}{2} \eta(\gamma, \tau, \sigma, \omega, \delta), \quad \eta > 0,
\label{eq:Phi_solution}
\end{equation}
where $\eta$ depends on all parameters but is strictly positive, proof is in the appendix of \cite{mandal2013refrigerator}. Consequently, the sign of $\Phi$ is determined by the difference between the incoming bias $\delta$ and the thermal parameter $\epsilon$. When $\delta > \epsilon$, the tape bias wins and we get refrigeration. When $\delta < \epsilon$, the temperature gradient wins and the demon acts as an eraser, leading to a $\delta'$ closer to $\epsilon$ than $\delta$ \cite{mandal2013refrigerator}.


To take into account the second law,  we look at the modified Clausius inequality
\begin{equation}
Q_{c\to h}(\beta_H - \beta_C) + \Delta S_B \geq 0,
\label{eq:second_law}
\end{equation}
where $\Delta S_B = S(\delta') - S(\delta)$ is the change in Shannon entropy per bit, with
\begin{equation}
S(\delta) = -\frac{1-\delta}{2}\ln\frac{1-\delta}{2} - \frac{1+\delta}{2}\ln\frac{1+\delta}{2}.
\end{equation}
This states that the thermodynamic entropy decrease (first term, negative when heat flows up the gradient) must be compensated by an increase in information entropy.


\subsection{Numerical Implementation}

Here, it is important to note that the original paper \cite{mandal2013refrigerator} does not contain a numerical implementation of the problem; rather, its main achievement is the analytic solution of this setup. This enables us to cross-check our numerical implementation with the analytical solution. 

\subsubsection{Gillespie Method}
We employ the stochastic simulation algorithm (SSA) of Gillespie~\cite{gillespie1976} inside each interaction window of duration \(\tau\), during which a single bit interacts with the demon and the thermal reservoirs.\footnote{Initially, I attempted to simulate multiple extremely small $dt$ windows at each $\tau$; however, this proved to be less reliable and significantly more computationally expensive. As such, I decided to only use the Gillespie Method going forward.}

For our joint state \(s\in\{0u,0d,1u,1d\}\), the total exit rate is
\[
r_{\text{out}}(s)=
\begin{cases}
\gamma(1+\sigma), & s=0u,\\[2pt]
\gamma(1-\sigma)+(1-\omega), & s=0d,\\[2pt]
\gamma(1+\sigma)+(1+\omega), & s=1u,\\[2pt]
\gamma(1-\sigma), & s=1d~.
\end{cases}
\] Thanks to Eq.~ \ref{eq:hot_rates} \& \ref{eq:cold_rates}


Given the current joint state \(s\) inside the window, let \(\{r_j\}\) be the rates of the reactions that can fire from \(s\) (one or two transition channels, depending on \(s\)). The SSA's waiting time is
\[
dt \sim \mathrm{Exp}\!\big(r_{\text{out}}(s)\big),\quad\text{i.e.}\quad dt=-\frac{\ln u_1}{r_{\text{out}}(s)},\; u_1\sim\mathcal U(0,1),
\]
and, conditional on a jump, picks channel \(j\) with probability \(r_j/r_{\text{out}}(s)\) using a second uniform \(u_2\).


For each bit–demon encounter:
\begin{enumerate}
\item Initialize \(t\leftarrow 0\) with incoming bit \(b\in\{0,1\}\) (drawn from \(p_0,p_1\)) and demon state \(d\in\{u,d\}\). Set joint state \(s=(b,d)\).
\item \textbf{While} \(t<\tau\):
  \begin{enumerate}
  \item Compute all enabled rates \(\{r_j\}\) from \(s\) and their sum \(r_{\text{out}}(s)\).
  \item Draw \(dt=-\ln u_1/r_{\text{out}}(s)\). If \(t+dt\ge\tau\): set \(t\leftarrow\tau\) and \textbf{break} (no jump occurs in this window).
  \item Otherwise, select the firing channel \(j\) by finding the smallest index with
  \[
  \sum_{k\le j-1} r_k < u_2\, r_{\text{out}}(s) \le \sum_{k\le j} r_k,\qquad u_2\sim\mathcal U(0,1).
  \]
  \item Update the joint state \(s\) according to \(j\), advance time \(t\leftarrow t+dt\).
 \end{enumerate}
\end{enumerate}



Additionally, for our numerical implementation, it is more convenient to make use of the dimensionless parameters $\sigma$, $\omega$, and $\epsilon$ going forward, and focus on the output parameter $\Phi$. $\Phi$ defines the \emph{net} number of $0\to1$ flips per bit, and each  $0\to1$ flip is an action against the thermal gradient, a refrigeration, and vice versa is also true for the eraser case ($\Phi<0$). Thus, it quantifies how close the system is to a perfect refrigerator; if $\Phi = 1$, all bit flips are $ 0 \ to 1$, which would mean that the heat flow is only from the cold reservoir to the hot reservoir. However, this would only be possible with $\epsilon = 0$,(See Eq. \ref{eq:Phi_solution}) which is frankly a boring system with $\sigma = \omega \rightarrow T_H = T_C$. Which is why our aim in this project is to get as close as possible to $\Phi = 1$ without changing the temperature differences or $\Delta E$. 

\subsubsection{Results}
With that out of the way, I first attempted to reproduce the phase diagram presented in the original paper using the aforementioned method.
\begin{figure}[h]
  \centering
  \begin{subfigure}{0.48\linewidth}
    \includegraphics[width=\linewidth]{phase_digs/phasse_dig_1.png}
    \caption{}\label{fig:p1}
  \end{subfigure}\hfill
  \begin{subfigure}{0.48\linewidth}
    \includegraphics[width=\linewidth]{phase_digs/phase_dig_2.png}
    \caption{}\label{fig:p2}
  \end{subfigure}

  \vspace{0.6em}

  \begin{subfigure}{0.48\linewidth}
    \includegraphics[width=\linewidth]{phase_digs/phase_dig_3.png}
    \caption{}\label{fig:p3}
  \end{subfigure}\hfill
  \begin{subfigure}{0.48\linewidth}
    \includegraphics[width=\linewidth]{phase_digs/Phase_Original.png}
    \caption{}\label{fig:p4}
  \end{subfigure}

  \caption{a, b, and c are my simulation results, while d is the exact solution in the paper. Black lines represent $\Delta S_B = 0$. The values on the lines in image d represent $\tau$ values, and plots a, b, c represent $\tau = 1.0, 4.0, 10000$ respectively, with blue areas acting as erasers and red parts representing refrigeration. Also all plots have $\omega = 1/2, \gamma = 0.1$}
  \label{fig:pgrid}
\end{figure}
\footnote{$n=2$ present throughout this section's plots represent the number of demon states}
As shown in FIG. \ref{fig:pgrid}, the simulation results are consistent with the phase diagram presented in the original paper (FIG. \ref{fig:p4}). The next part that I wanted to check was the $\gamma \to \infty$ limit, as the original paper states that in that limit, the expression for $\Phi$ simplifies to:
\begin{equation}
    \Phi = \frac{\delta - \epsilon}{2} [1-e^{-(1-\sigma \omega )\tau}].
\end{equation} \cite{mandal2013refrigerator}

Thus, we can check if we approach this limit with high gamma. 
\begin{figure}
  \centering
    \begin{subfigure}{0.9\linewidth}
    \includegraphics[width=\linewidth]{gamma/gamma_vs_phi_higher_tau_3.png}
    \caption{}\label{fig:ghtau3}
  \end{subfigure}\hfill
  \begin{subfigure}{0.9\linewidth}
    \includegraphics[width=\linewidth]{gamma/gamma_vs_phi_higher_tau_1.png}
    \caption{}\label{fig:ghtau1}
  \end{subfigure}\hfill
  \begin{subfigure}{0.9\linewidth}
    \includegraphics[width=\linewidth]{gamma/gamma_vs_phi_higher_tau_2.png}
    \caption{}\label{fig:ghtau2}
  \end{subfigure}\hfill

  \caption{Here we go through gamma values [1-100] and plot out the $\Phi$ values. We observe the desired behavior and approach the theoretical limit very fast, thanks to the high  $\tau = 5.0$. Note the different $\sigma$ and $\omega$ values for each plot, which are mainly for confirmation.}
  \label{fig:gamma1}
\end{figure}
\begin{figure}
  \centering
    \begin{subfigure}{0.9\linewidth}
    \includegraphics[width=\linewidth]{gamma/gamma_vs_phi_sigmaomega_3.png}
    \caption{}\label{fig:ghtau6}
  \end{subfigure}\hfill
  \begin{subfigure}{0.9\linewidth}
    \includegraphics[width=\linewidth]{gamma/gamma_vs_phi_sigmaomega_1.png}
    \caption{}\label{fig:ghtau4}
  \end{subfigure}\hfill
  \begin{subfigure}{0.9\linewidth}
    \includegraphics[width=\linewidth]{gamma/gamma_vs_phi_sigmaomega_2.png}
    \caption{}\label{fig:ghtau5}
  \end{subfigure}\hfill

  \caption{Here we go through gamma values [1-100] and plot out the $\Phi$ values but approach the theoretical limit slower than the $\tau = 5.0$ case, since here $\tau = 1.0$. This is a pattern we will later see again.}
  \label{fig:gamma2}
\end{figure}

Figures \ref{fig:gamma1} and \ref{fig:gamma2} show that the numerical simulation gets within the theoretical limit with each parameter set. The other pattern we observe is that there might be a positive correlation with $\tau$ and $\Phi$. 
\begin{figure}[H]
  \centering
  \begin{subfigure}{0.9\linewidth}
    \includegraphics[width=\linewidth]{phivstau/phi_vs_tau_1.png}
    \caption{}\label{fig:phivtau1}
  \end{subfigure}\hfill
  \begin{subfigure}{0.9\linewidth}
    \includegraphics[width=\linewidth]{phivstau/phi_vs_tau_2.png}
    \caption{}\label{fig:phivtau2}
  \end{subfigure}\hfill
  \begin{subfigure}{0.9\linewidth}
    \includegraphics[width=\linewidth]{phivstau/phi_vs_tau_3.png}
    \caption{}\label{fig:phivtau3}
  \end{subfigure}\hfill
  \caption{Here we sweep through $\tau$ values between 1.0- 100.0 and see both how increasing $\tau$ leads to increasing $\Phi$. The interesting point is that we arrive at the same theoretical ceiling as before, without setting $\gamma \to \infty$, $\gamma = 1.0$ throughout a, b, c. The physical parameters determine the limit again.}
  \label{fig:phivtaugrid}
\end{figure}

\begin{figure}[H]
  \centering
  \begin{subfigure}{0.9\linewidth}
    \includegraphics[width=\linewidth]{N/length_vs_phi_1.png}
    \caption{}\label{fig:Nvsphi1}
  \end{subfigure}\hfill
  \begin{subfigure}{0.9\linewidth}
    \includegraphics[width=\linewidth]{N/length_vs_phi_2.png}
    \caption{}\label{fig:Nvsphi2}
  \end{subfigure}\hfill
  \caption{Here we go through N values from 200 to 80000. I split the plots so that the latter part is clearer to the eye, and we can see that the $\Phi$ value stabilizes around $N=10000$}
  \label{fig:Nvsphigrid}
\end{figure}

Looking at the FIG. \ref{fig:phivtaugrid} shows that we reach the same theoretical limits as in FIG. \ref{fig:gamma1}-\ref{fig:gamma2} when we check for the same $\omega$ and $\sigma$ values. This reveals that $\tau$ and $\gamma$ can lead to an increase in $\Phi$ up to a point determined by $\omega$ and $\sigma$ values. Lastly, the relationship between the tape's length and $\Phi$ can also be seen in FIG. \ref{fig:Nvsphigrid}, which is kind of expected and does not show any reason to go for a bigger $N$ than 10000. However, I chose a smaller $N=4000$ and sacrificed accuracy for speed, as the main aim of this section was primarily to observe patterns.

To conclude this lengthy section, although we obtained interesting results by varying $\tau$ and $\gamma$, the overall result is that we would need to redesign the system to achieve the aforementioned desired, more efficient refrigerator. 
 
\section{n-State Demon: Generalization to Energy Ladders}
\label{sec:n_state}

The two-state demon has only one energy quantum $\Delta E$ to mediate the transition. A natural question is: would a demon with a different internal structure—more energy levels—perform better? Can we get a system that gets as close to a perfect refrigerator as physically can?


We can generalize the demon to $n$ states arranged on an energy ladder: $E_0 < E_1 < \cdots < E_{n-1}$ with spacing $\Delta E_k = E_{k+1} - E_k$.  The joint system now has $2n$ states. The key design choice is how to couple the ladder to bit flips. Generalizing the previous interactions fits our needs: a cooperative transition flips the bit from $0$ to $1$, i.e., $(b=0, s_k) \leftrightarrow (b=1, s_{k+1})$ for $k = 0, \ldots, n-2$. This preserves the rectification structure: you cannot flip to a $1$ bit without raising the demon's energy, and you cannot flip to a $0$ bit without lowering it.


The hot and cold transition channels become:

Hot channel: Transitions $s_i \leftrightarrow s_j$ within each bit sector ($b=0$ or $b=1$) obey detailed balance at $T_H$:
\begin{equation}
\frac{R_{s_i \to s_{i+1}}}{R_{s_{i+1} \to s_i}} = e^{-\beta_H (E_{i+1} - E_i)}.
\label{eq:firt}
\end{equation}
 I use the same parametrization as before: $R_{s_k \to s_{k+1}} = \gamma(1 - \sigma_k)$ and $R_{s_{k+1} \to s_k} = \gamma(1 + \sigma_k)$, where now $\sigma_k = \tanh(\beta_H \Delta E_k/2)$ depends on the local spacing.

Cold channel: Cooperative transitions $(b=0, s_k) \leftrightarrow (b=1, s_{k+1})$ obey detailed balance at $T_C$:
\begin{equation}
\frac{R_{0s_k \to 1s_{k+1}}}{R_{1s_{k+1} \to 0s_k}} = e^{-\beta_C \Delta E_k},
\label{eq:secnt}
\end{equation}
parametrized as $R_{0s_k \to 1s_{k+1}} = 1 - \omega_k$ and $R_{1s_{k+1} \to 0s_k} = 1 + \omega_k$ with $\omega_k = \tanh(\beta_C \Delta E_k/2)$.

For a uniform ladder ($\Delta E_k = \Delta E/(n-1)$ for all $k$), the parameters $\sigma_k$ and $\omega_k$ become constant, recovering the two-state $\epsilon = (\omega - \sigma)/(1 - \omega\sigma)$ as the effective asymmetry. Note that plots below will not have $\sigma_k$ and $\omega_k$. The modified Clausius inequality \eqref{eq:second_law} continues to hold, though the functional form of $\eta$  now depends on $N$ and the ladder profile, and may be unattainable.

\subsection{Numerical Implementation}

We can reuse the previous SSA method by tracking the $2N$-state joint system and its corresponding transition rates in equations \ref{eq:firt} and \ref{eq:secnt}.


\subsubsection{Results: Number of Demon States vs $\Phi$}

\begin{figure}[H]
    \centering
    \includegraphics[width=0.9\linewidth]{demonn/Demon_Result.png}
    \caption{Here we have $\omega=0.7$, $\sigma=0.2$. $\tau = 10.0$ to hit the aforementioned limit earlier. We see a clear rise in $\Phi$ as we increase the size of the energy ladder.}
    \label{fig:demonRes}
\end{figure}

\begin{figure}[H]
    \centering
    \includegraphics[width=0.9\linewidth]{demonn/demon_n_vs_phi_2.png}
    \caption{For this one, we continue to increase the size of the ladder, but not much change happens after $n=20$.}
    \label{fig:demonRes2}
\end{figure}
\begin{figure}[H]
    \centering
    \includegraphics[width=0.75\linewidth]{demonn/Low_DeltaE.png}
    \caption{We have a much lower $\Delta E = 0.05$, which is the energy difference between states for a demon with 21 states.}
    \label{fig:lowdeltaE}
\end{figure}
We can see in FIG. \ref{fig:demonRes} that our design has worked, at least from a computational perspective, and we have gotten a much higher $\Phi$ without changing $\omega$ or $\sigma$. One might argue that this is largely due to the fact that the energy difference between states is significantly lower. However, we can see in FIG. \ref{fig:lowdeltaE} that this is not the case, lower $\Delta E$ does not make that much of a difference, comparing FIG. \ref{fig:phivtau2} and \ref{fig:lowdeltaE} shows exactly this. 


\section{Conclusion}
\label{sec:conclusion}
We reproduced the behavior of Mandal \emph{et al.}\ \cite{mandal2013refrigerator} using a Gillespie simulation. The numerical phase diagrams match the exact solution, a sign that the modified Clausius bound is respected we are on the right track. We also verified the fast–demon limiting form
\begin{equation}
  \Phi=\frac{\delta-\epsilon}{2}\left[1-e^{-(1-\sigma\omega)\tau}\right],
\end{equation}
and observed that increasing either \(\gamma\) or \(\tau\) raises \(\Phi\) only up to a ceiling fixed by \(\sigma\) and \(\omega\). Testing on $N$(tape length) showed that moderate tape lengths are sufficient for stable estimates of \(\Phi\).

Generalizing the demon to an \(n\)-level energy ladder systematically boosts \(\Phi\) at fixed bath parameters and total \(\Delta E\), because cooperative bit flips can be effected through smaller energetic steps. The gain exhibits diminishing returns beyond modest ladder sizes (here, \(n\!\gtrsim\!20\)), and the same thermodynamic trade-off between heat flow and information processing is present of course: refrigeration \((Q_{c\to h}>0)\) should be accompanied by writing information, while erasure requires heat to flow down the gradient.

Overall, the extended model provides a straightforward approach to more efficient autonomous 'refrigerators' without altering the reservoirs: restructure the internal state design.

\section{Future Work}
\label{sec:future}

I have not been able to fully examine the analytic part of the original problem, which is why my number one priority is to get to a satisfactory level of familiarity with the solution and then, to get as close as possible to an analytic treatment for the $n$ level demon case, which will include lots of optimizations.

 In short, next steps are (i) an analytic treatment of first the 2 state case and then the \(N\)-state case (uniform and non-uniform \(\Delta E_k\)), (ii) optimizing ladders and rates for maximal \(\Phi\) at fixed \((T_H, T_C,\Delta E)\), and (iii) checking designs with $n>2$ bit states. 


\bibliographystyle{unsrt}
\bibliography{refs}
\appendix
\section{Code}\label{app:deriv}
All the code relevant to this project can be found at \href{https://github.com/erdemydt/491_Project}{link}; the most useful files are located in the phase\_4 folder. Simulation.py could be and was used to generate a bunch of plots relevant to the topic.
\end{document}
